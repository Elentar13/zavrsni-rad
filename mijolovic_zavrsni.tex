\documentclass[a4paper,12pt,oneside]{memoir}

%documentation search
%http://texdoc.net/

%matematički paketi

\usepackage[intlimits]{amsmath}     %omogućava postavljanje granica integrala u formulama
\usepackage{amsthm}         %matematički teoremi, leme i sl.
\usepackage{siunitx}        %podrška za korištenje SI sustava mjernih jedinica

%encoding fontova i jezika
\usepackage[croatian]{babel}
\usepackage[utf8x]{inputenc}        %encoding inputa
\usepackage[enc=utf8]{hrlatex}
\usepackage[T1]{fontenc}        %encoding fontova koji je prikazan u PDF-u
\usepackage{amsfonts}
\usepackage{dsfont}

\usepackage[fixlanguage]{babelbib}
\selectbiblanguage{croatian}
\OnehalfSpacing

%\usepackage[datetime2-croatian]{datetime2}

%paketi tablica, naslova, poglavlja i sl.
\usepackage[thinlines]{easytable}
\usepackage{tocloft}        %upravljanje izgledom tablice sadržaja
\usepackage{pdfpages}       %integracija eksternih PDF-ova
\usepackage{booktabs}       %koristi se za formatiranje tablica sukladno standardu za znanstvene radove i članke
\usepackage{indentfirst}     %dodaje tab za svaku prvu rečenicu odlomka
\usepackage{subcaption}     %koristi se za podnaslove slika, formi i sl.
\usepackage[font=it]{caption}
\captionsetup[table]{position=above}
\captionsetup[figure]{position=below}
\captionsetup{labelsep=period}

\usepackage[hidelinks]{hyperref}        %podrška za integraciju hyperlinkova
\urlstyle{same}

\usepackage{float}
%grafički paketi
% \usepackage{pgfcore}
% \usepgflibrary{datavisualization.formats.functions}
\usepackage{graphicx}
\usepackage{pgfmath}
\usepackage{pgfplots}
\pgfplotsset{compat=1.17}
\pgfplotsset{
    standard/.style={
        width = 0.9\linewidth,
        semithick,
        tick style={major tick length=4pt,semithick,black},
        every axis plot post/.style={mark options={fill=black}},
        separate axis lines,
        axis x line*=bottom,
        axis x line shift=5pt,
        %xlabel shift=5pt,
        axis y line*=left,
        axis y line shift=5pt,
        ylabel shift=5pt,
        xtick align = outside,
        ytick align = outside,
        xlabel near ticks,
        ylabel near ticks,
        % xmin = -2, xmax = 2,
        % ymin = -1, ymax = 1,
        grid = both
    }
}
\usepackage{tikz}
\usetikzlibrary{datavisualization}
\usetikzlibrary{datavisualization.formats.functions}
\usetikzlibrary{arrows,automata,patterns,positioning}
\tikzstyle{int}=[draw, fill=gray!20, minimum size=2em]
\tikzstyle{init} = [pin edge={to-,thin,black}]
\usepackage{circuitikz}
\newcommand{\pgfmathparseFPU}[1]{\begingroup%
\pgfkeys{/pgf/fpu,/pgf/fpu/output format=fixed}%
\pgfmathparse{#1}%
\pgfmathsmuggle\pgfmathresult\endgroup}


%misc. paketi

\usepackage{soul} %žuti marker
\usepackage{times}

%formatiranje dokumenta
\pagestyle{myheadings}
\setulmarginsandblock{2.5cm}{2.5cm}{*}
\setlrmarginsandblock{2.5cm}{2cm}{*}
\checkandfixthelayout

\setlength{\parskip}{6pt} %razmak između odlomaka

\usepackage{titlesec}   %nadomješta LaTeX makroe za naslove, odlonke, itd.

\titleformat{\chapter}
{\normalfont\fontsize{14}{14}\bfseries}
{\thechapter}
{1em}
{}
\titlespacing{\chapter}{0pt}{*4}{*1}
\titleformat{\section}
{\normalfont\fontsize{12}{14}\bfseries}
{\thesection}
{1em}
{}
\titlespacing{\section}{0pt}{*4}{*1}
\setsecnumdepth{subsection}
\maxtocdepth{subsection}
\titleformat{\subsection}
{\normalfont\fontsize{12}{14}}
{\thesubsection}
{1em}
{}
\titlespacing{\subsection}{0pt}{*4}{*1}

%formatiranje dokumenta - dodavanje točaka u naslove

\renewcommand{\thechapter}{\arabic{chapter}.}
\renewcommand{\thesection}{\thechapter\arabic{section}.}
\renewcommand{\thesubsection}{\thesection\arabic{subsection}.}

%formatiranje dokumenta - promjena fonta naslova "Sadržaj"

\renewcommand\printtoctitle[1]{\normalfont\fontsize{14}{14}\bfseries #1}

%formatiranje dokumenta - promjena riječi "Dodatak" u sadržaju

\renewcommand*{\cftappendixname}{Dodatak\space}

%formatiranje dokumenta - numeriranja tablica i slika



\renewcommand{\thetable}{\thechapter\arabic{table}}
\renewcommand{\thefigure}{\thechapter\arabic{figure}}

%teoremi

\newtheorem{theorem}{Teorem}[chapter]
\newtheorem{korolar}{Korolar}[chapter]
\newtheorem{definition}{Definicija}[chapter]
\newtheorem{lema}{Lema}[chapter]
\newtheorem{prop}{Propozicija}[chapter]
\newtheorem{primjer}{Primjer}[chapter]

%formule

\newcommand{\fourierovred}{f(x)= \frac{a_0}{2}+\sum_{n=1}^\infty (a_n cos(nx)+b_n sin(nx))}
\newcommand{\nasavjerojatnost}{p(t)= \frac{e^{a+b\cdot t}}{1+e^{a+b\cdot t}}}
\newcommand{\kauzalni}{y[n]=\frac{1}{M}\displaystyle\sum_{k=0}^{M-1 }x[n-k]}
\newcommand{\nekauzalni}{y[n]=\displaystyle\sum_{k=-\infty}^{n}x[n-k]}

%primjer parametriziranih TikZ grafova
%http://www.texample.net/tikz/examples/parameterized-plots/

%primjer normalne razdiobe
%http://www.texample.net/tikz/examples/tikzdevice-demo/

%formatiranje dokumenta - numeriranja jednadžbi, definicija i teorema

\renewcommand\theequation{\thechapter\arabic{equation}}
\renewcommand\thetheorem{\thechapter\arabic{theorem}}
\renewcommand\thedefinition{\thechapter\arabic{definition}}
\renewcommand\thekorolar{\thechapter\arabic{korolar}}
\renewcommand\theprimjer{\thechapter\arabic{primjer}}


\author{Denis Mijolović} 

\begin{document}
    \begin{titlingpage}
        \begin{center}\linespread{1.8}\fontsize{16}{16}
            \normalfont{
                SVEUČILIŠTE U RIJECI
            }\\
            {\textbf{
                TEHNIČKI FAKULTET
                }
            }\\
        \fontsize{14}{14}\normalfont{
            Preddiplomski sveučilišni studij elektrotehnike
        }            
        \end{center}
        \vspace{6cm}
        \begin{center}\linespread{1.8}
            \fontsize{14}{14}\normalfont{
                Završni rad
            }\\
            \fontsize{16}{16}{\textbf{
                AUTOREGRESIJSKI MODELI U OBRADI SIGNALA
                }
            }
        \end{center}
        \vspace{6cm}
        \vfill\noindent
        \begin{tabular}[t]{l@{}}
            \fontsize{14}{14}\normalfont{
                Rijeka, \today
            }
        \end{tabular}
        \hfill
        \begin{tabular}[t]{l@{}}
            \fontsize{14}{14}\normalfont{
                Denis Mijolović
            }\\
            \fontsize{14}{14}\normalfont{
                0069066432
            }
        \end{tabular}
        \thispagestyle{empty}
        \newpage
            \begin{center}\linespread{1.8}
                \fontsize{16}{16}\normalfont{
                    SVEUČILIŠTE U RIJECI
                }\\
                {\textbf{
                    TEHNIČKI FAKULTET
                    }
                }\\
                \fontsize{14}{14}\normalfont{
                    Preddiplomski sveučilišni studij elektrotehnike
                }
            \end{center}
            \vspace{6cm}
            \begin{center}\linespread{1.8}
                \fontsize{14}{14}\normalfont{
                    Završni rad
                }\\
                \fontsize{16}{16}{\textbf{
                    AUTOREGRESIJSKI MODELI U OBRADI SIGNALA
                    }
                }\\
                \fontsize{14}{14}\normalfont{
                Mentor: doc. dr. sc. Ivan Dražić
            }\\
            \fontsize{14}{14}\normalfont{
                Komentor: prof. dr. sc. Viktor Sučić
            }
            \end{center}
        \vfill\noindent
        \begin{tabular}[t]{@{}l}
            \fontsize{14}{14}\normalfont{
                Rijeka, \today
            }        
        \end{tabular}
        \hfill
        \begin{tabular}[t]{@{}l}
            \fontsize{14}{14}\normalfont{
                Denis Mijolović
            }\\
            \fontsize{14}{14}\normalfont{
                0069066432
            }
        \end{tabular}
        \thispagestyle{empty}
        \newpage
            \thispagestyle{empty}
            Na mjesto ovo stranice je potrebno umetnunti izvornik zadatka
        \newpage
            \thispagestyle{empty}
                \begin{center}\linespread{1.8}\fontsize{16}{16}
                    {\textbf{
                        IZJAVA
                        }
                    }
                \end{center}
            \vspace{5cm}
            Sukladno članku 8. Pravilnika o završnom radu, završnom ispitu i završetku prediplomskih sveučilišnih studija/stručnih studija
			Tehničkog fakulteta Sveučilišta u Rijeci od 1. veljače 2020., izjavljujem da sam samostalno izradio završni rad
            prema zadatku preuzetom dana 18. ožujka 2019. godine.
            \vspace{3cm}


            \begin{tabular}[t]{@{}l}
                \normalfont{
                    Rijeka, \today
                }
            \end{tabular}
            \hfill
            \begin{tabular}[t]{@{}l} 
                \fontsize{14}{14}\hrulefill\\
                \fontsize{12}{12}\normalfont{
                    \phantom{Raz}Denis Mijolović\phantom{Raz}
                }
            \end{tabular}
            \newpage
                \thispagestyle{empty}
                $ $
                    \vspace{5cm}
                    \textit{
                        \\
                        Lorem ipsum dolor sit amet, consectetur adipiscing elit. Mauris et ante a mauris consequat tempor. Donec vulputate, nulla pharetra finibus finibus, ipsum ante molestie mauris, in tincidunt libero augue sit amet arcu. In hac habitasse platea dictumst. Nulla sagittis ante ac augue volutpat cursus. Ut gravida odio malesuada, vestibulum mi at, dignissim leo. Morbi semper finibus est, id maximus mauris auctor eget.
                    }
    \end{titlingpage}
    \begin{KeepFromToc}
        \tableofcontents
    \end{KeepFromToc}
    \chapter{Uvod}
        \textbf{(Definicija podataka i informacija na zanimljiv način...ako je takvo što moguće)}

        Vratimo li se u prošlost, važnost podataka i njihova inerpretacija u svrhu stvaranja informacija bila je neminovna za civilizacijske strukture koje danas poznajemo. Od početka razvoja primitnivnih tehnologija i njezinoga korištenja za optimizaciju životnih procesa, do razvoja trgovine i financijskih instrumenata; čovjek je sve efikasnije koračao prema evolucijskom vrhu uspješnom utilizacijom svoje okoline. Daljnjim razvojem, čovjek je nadišao barijeru vlastite memorije za obradu podataka, te se počeo približavati točki zasićenja - uslijed koje se javila potreba za stvaranjem boljega i efikasnijega načina obrade podataka i njihove pretvorbe u sažetiju i jednako kvalitetnu informaciju.

        Kao i kod većine velikih znanstvenih otkrića koja su obilježila svoje razdoblje kao pretkretnice stoljeća u ratnim okolnostima, tako je i tijekom Drugog svjetskog rata otkriven novi izum kao odgovor na Enigmu - stroj koji je kodirao njemačke vojne i diplomatske poruke te čija se šifra smatrala nerješivom. Uzevši u obzir širenje nacističke Njemačke i prijetnju koja je prijetila ljudskim slobodama, javila se jasna potreba za stvaranjem rješenja koje će nadići čovjekova fizička i psihička ograničenja te biti u mogućnosti procesuirati veliku količinu podataka u vrlo kratkom vremenu -- radeći neprestano i ne osjećajući umor te, konzekventno, ne stvarajući greške. Imajući to na umu, Englezi su tada izumili \textit{Colossus} -- uređaj koji je pomogao kriptoanalitičarima da dešifriraju njemačke poruke i dao im stratešku prednost pri prognozorianju daljnih koraka tijekom ratnog razdoblja. Taj uređaj se u današnje vrijeme smatra pretečom prvoga računala kakvim ga danas poznajemo.

        Baš kao i tada, te kroz cijelu ljudsku povijest, ispravna interpretacija podataka u informaciju koju krajnji korisnik može iskoristiti u svrhu kvalitetnijeg donošenja važnih odluka, iskazala se kao značajan faktor u očuvanju civilizacijske i sveopće stabilnosti -- stanja koje je najpotentnije za budući razvoj, a time i blagostanje. Kako je civilizacija težila ka tehnološkom društvu, tako je eksponencijalno rasla i količina podataka koju je bilo potrebno interpretirati; te se javila potreba za sve preciznijim prognoziranjem budućega ponašanja sustava kako ne bi došlo do destrukcije vitalnih procesa i porasta volatilnosti sustava - a samim time i povećanjem vjerojatnosti za njegovim urušavanjem.

        Slijedno tome, važnost prognoziranja proizlazi iz dvije osnovne činjenice:
        \begin{enumerate}
            \item budućnost je nepredvidiva
            \item poduzete akcije u vrijeme donošenja odluke vrlo često nemaju trenutne posljedice sve do određenoga trenutka u budućnosti
        \end{enumerate}


        Imajući na umu prethodno navedene činjenice, slijedno je zaključiti da precizne prognoze budućih događaja pospješuju efikasnost postupka donošenja odluka. Štoviše, većina odluka od posebnog poslovnog ili političkog značaja imaju nužan uvjet pozitivnih prognostičkih indikatora prezentiranih kroz studije izvedivnosti ili ispitivanja javnoga mnijenja -- čija je zadaća što preciznije prognozirati buduću potražnju za proizvodima i uslugama koje su predmet proizvodnih, industrijskih ili inih procesa. U konkretnijem, mikroekonomskom smislu, dalo bi se konstatirati da je prognoziranje implementirano u gotovo sve odluke koje su dio naše svakodnevice: poduzeće će početi izgradnju nove proizvodne jedinice kako bi osiguralo rastuću potražnju na tržištu; zaposleni radnici početi će štedjeti kako bi si mogli priuštiti godišni odmor ili kupnju vrijednosnica s ciljem ostvarivanja prava na buduću kapitalnu dobit; rektor sveučilišta donesti će odluku o otvaranju novog studijskog programa kako bi povećao broj studenata i uskladio broj obrazovanih stručnjaka s budućim tržišnim potrebama. Ukratko, svaki od navedenih procesa zahtjeva inicijalnu percepciju utjecaja donešenih odluka na povećanje vjerojatnosti ostvarivanja zadanih ciljeva -- i to prije donošenja samih odluka.


        Ostanemo li na mikroekonomskoj razini, možemo reći da je poslovanje poduzeća zahvaćeno trijema faktorima: makroekonomskim prilikama, industrijskim pokazateljima i samim uređenjem poduzeća. Obzirom da direktor takvoga poduzeća najčešće ima mogućnost direktnoga djelovanja jedino na poslijednja dva faktora, od iznimne je važnosti da bude što bolje informiran o vanjskim utjecajima i trendovima na koje ne može izravno djelovati -- poput donošenja novih zakonskih regulativa ili razvoja globalne virusne pandemije. Važnost ispravnog prognoziranja čimbenika na koje se nema direktan utjecaj od iznimne je važnosti za većinu projekta iznimno velike vrijednosti. Tako, na prijmer, projekti s velikim početnim ulaganjima poput izgradnje novih jedinica za opskrbu električnom energijom, mogu imati vrijeme izrade studije izvedivosti u trajanju od 10 ili više godina prije nego što li se započnu radovi. Budući da opskrba električnom energijom ovisi o stupnju razvijenosti regije ili područja, potrebno je izraditi valjane prognoze koje će pomoći u procjeni trenutne, kratkoročne i dugoročne isplativosti novih opskrbnih jedinica za opskrbljivača, ostale projektne partnere, te za lokalni i regijalni razvoj. U tom slučaju, prisutnost raznovrsnih rizika utječe na sve dionike projekta -- od nacionalne, regionalne ili lokalne uprave ili samouprave; do svih industrijskih poduzeća koja se nalaze na tom području. Ti rizici mogu biti smanjenje likvidnosti tržišta, kreditni status\footnote{Usporedba potraživanja i dugovanja nekoga poduzeća koje se dalje koriste za utvrđivanje kreditnog rejtinga.}, kamatni rizik\footnote{Moguće smanjenje vrijednosti financijskih instrumenata javno dostupnoga poduzeća uslijed promjene razine kamatnih stopa na tržištu.},valutni rizik\footnote{Osjetljivost financijskih instrumenata na fluktuacije tečajeva stranih valuta. Značajno za multinacionalna poduzeća s viševalutnim potraživanjima i dugovanjima.}, strateški rizik\footnote{Rizik za prihode ili kapital koji nastaje kao posljedica neadekvatnih poslovnih odluka ili nepravilnog vođenja poduzeća.}, stanje tržišta rada i još mnogi drugi. Pritom je važno napomenti važnost osjetljivosti modela prognoziranja na fluktuacije u svakoj stavci procjene ukupnoga rizika, tj. isti prognostički model se ne bi trebao koristiti za prognoziranje u različitim zakonskim i regulativnim okvirima -- što pridonosi kompleksnosti izrade samoga modela i porastu troškova izrade kvalitetne studije.


        U navedenim primjerima nastojali smo nešto konkretnije opisati važnost prognoziranja kroz opće primjere iz prakse. Kako bismo zaokružili sadržaj ovoga poglavlja, osvrnuti ćemo se na stavrni primjer jedne od najutjecajnijih nesreća koje su obilježile prošlo stoljeće --  no ovoga puta fokusirajući se na ispravnoj interpretaciji prognostičkih procesa i važnosti najnasumičnijeg čimbenika u cijelome procesu odlučivanja, tj. čovjeka.


        28. siječnja 1986. godine, svemirska letjelica \textit{Challanger} eksplodirala je otprilike 70 sekundi nakon polijetanja iz svemirskog centra John F. Kennedy u Floridi. Sedmero astronauta je poginulo, a svemirska letjelica u potpunosti je uništena. Uzrok nesreće bila je eksplozija spremnika za raketno gorivo, prouzrokovana zapaljenjem plina u pomoćnim raketama za uzlijetanje.

        Pomoćne rakete za uzlijetanje bile su predmet zabrinutnosti još pri samim počecima razvoja aeronautike. Proizvodnja istih realizira se spajanjem više manjih segmenata uz veliki broj spojišta koji moraju zadovoljavati niz projektom predodređenih uvjeta -- među kojima je postizanje nepropusnosti spojišta brtvljenjem. U ovom konkretnom slučaju, korištene su meke profilne brtve s elastičnim deformacijama (također poznate kao o-prsteni) uz sloj ljepila. Prilikom zapaljenja raketnog motora, u sustavu se stvaraju visoke temperature i visoki tlak -- što utječe na otapanje ljepila i posljedično habanje brtve, te propuštanja eksplozivnog medija.

        Nakon eksplozije, istražni stožer objavio je izvješće u kojemu su navedene okolnosti koje su dovele do uzroka nesreće. Utvrđeno je kako je prethodne večeri, 27. siječnja, donesena odluka o lansiranju letjelice sljedećega dana usprkos nepovoljnoj prognozi temperaturnih uvjeta određenih radnom temperaturom deklariranom od strane proizvođaća pomoćnih raketa. Protivno apelu inženjera za obustavom lansiranja, odluka uprave bila je da će se lansiranje ipak provesti, što je rezultiralo kobnim posljedicama.


        Kako bi izradili model vjerojatnosti zakazivanja brtvi, koristiti ćemo podatke o učestalosti kvarova brtvi prilikom prethodnih lansiranja -- budući da je lansiranje \textit{Challanger}-a bilo 24. lansiranje aktualnog svemirskoga programa. Budući da su za svaku pomoćnu raketu korištena tri o-prstena te da su za dotadašnja lansiranja korištene dvije pomoćne rakete po lansiranju, razmatrati ćemo korištenje 6 o-prstena po lansiranju. Uslijed zavisnosti karakteristika brtve o temperaturi, razmotriti ćemo ovisnost zakazivanja brtve o temperaturi koline prilikom lansiranja. Slika \ref{fig:M1} prikazuje podatke o učestalost kvarova brtvi pri određenim temperaturama prilikom testnih lansiranja. Iz navedene uočavamo veću učestalost kvarova pri nižim temperaturuama -- što odgovara svojstvu amorfnih materijala, koji na temperaturama nižim od temperature prelaska u staklo imaju veću tvrdoću i lomljivost.

        %https://history.nasa.gov/rogersrep/v1p146.htm
        %DTIC_ADA171402 (page 151) --> temperature na page 136
        %F.M. Dekking, C. Kraaikamp, H.P. Lopuhaä, L.E. Meester-A Modern Introduction to Probability and Statistics (page 20)
        \begin{figure}[H]
            \centering
            \begin{tikzpicture}
                \draw[ultra thin, color=lightgray] (15,3) grid (0,0);
                \draw[-] (0,0) -- (15,0) node[pos=1, label={[below=0.8cm]center:$t/\textit{\textdegree C}$}] {} node[pos=0.5, align=center, label={[below=0.8cm]center:Temperatura spojišta}, label position=below] {};
                %\node[anchor=0.6cm, align=center] {$\text{tekst}$};
                \node[anchor=north] at (0,0) {$5$};
                \node[anchor=north] at (3,0) {$10$};
                \node[anchor=north] at (6,0) {$15$};
                \node[anchor=north] at (9,0) {$20$};
                \node[anchor=north] at (12,0) {$25$};
                \node[anchor=north] at (15,0) {$30$};
                %podijeliti celzijuse sa 1.67 kako bi se dobile koordinate na x-osi (popraviti cijeli tikz ispod ove linije). Oduzezi 3 jer je početna vrijednost 5 umjesto 0.
            
                \draw[-] (0,0) -- (0,3) node[pos=0.5, rotate=90, align=center, label={[rotate=90, left=0.8cm]center:Učestalost kvarova}, label position=above] {};
                \node[anchor=east] at (0,0) {$0$};
                \node[anchor=east] at (0,1) {$1$};
                \node[anchor=east] at (0,2) {$2$};
                \node[anchor=east] at (0,3) {$3$};

                \filldraw
                    (4,3) circle (2pt) node[align=center, above] {STS 51-C}
                    (5.317,1) circle (2pt) node[align=center, above] {41B}
                    (5.647,1) circle (2pt) node[align=center, below] {61C}
                    (7.311,1) circle (2pt) node[align=center, above] {41C}
                    (9.641,1.05) circle (2pt) node[align=center, above] {41D}
                    (9.641,0.95) circle (2pt) node[align=center, below] {STS-2}
                    (11.305,2) circle (2pt) node[align=center, above] {61A}

                    %successful launches
                    (8.311,0) circle (2pt)
                    (8.641,-0.1) circle (2pt)
                    (8.641,0) circle (2pt)
                    (8.641,0.1) circle (2pt)
                    (9,0) circle (2pt)
                    (9.311,0) circle (2pt)
                    (9.641,-0.05) circle (2pt)
                    (9.641,0.05) circle (2pt)
                    (10.305,0) circle (2pt)
                    (10.641,0) circle (2pt)
                    (11.305,0) circle (2pt)
                    (11.635,-0.05) circle (2pt)
                    (11.635,0.05) circle (2pt)
                    (12.305,0) circle (2pt)
                    (12.635,0) circle (2pt)
                    (12.970,0) circle (2pt)
                    (12.305,0) circle (2pt)
                    (13.299,0) circle (2pt);
                \draw[-] (8.25,0.15) -- (8.25,0.25) -- (13.35,0.25) -- (13.35,0.15);
                \draw[-{Latex[length=3mm]}] (13,1.25) node[rectangle, fill=white, draw, very thick, font=\small, align=center] {USPJEŠNA\\LANSIRANJA} -- (11,1.25) -- (11,0.25);
            \end{tikzpicture}
            \caption{Podaci o učestalosti kvarova brtvi prilikom testnih lansiranja koja su prethodila Challanger-u \cite{NASA}}
            \label{fig:M1}
        \end{figure}
        
        
        Model prikazan jednadžbom \eqref{eq:U1} reprezentira vjerojatnost kvara $p(t)$ pojedine brtve u ovisnosti o temperaturi spojišta $t$ prilikom lansiranja.
        
        \begin{equation}
            \nasavjerojatnost
            \label{eq:U1}
        \end{equation}

        gdje je:
        \begin{table}[H]
            \centering
            \begin{tabular*}{0.9\textwidth}{>{\bfseries}l p{13cm}}
                \textit{\textbf{t}} & \textit{temperatura brtve izražena u \textdegree F}\\
                \textit{\textbf{a,b}} & \textit{konstante dobivene statističkim modeliranjem metodom procjene maksimalne vrijednsosti}\\
            \end{tabular*}
        \end{table}
        % \indent\indent \textit{\textbf{t} \quad temperatura brtve izražena u \textdegree F}\\
        % \indent\indent \textit{\textbf{a, b} \quad konstante dobivene statističkim modeliranjem metodom procjene maksimalne vrijednsosti}
        

        Vrijednost konstanti $a$ i $b$ određene su podacima, tj. odabirom vrijednosti $a$ i $b$ tako da dobijemo što točniju aproksimaciju u okolini podatkovnih točaka prikazanih na slici \ref{fig:M1}. Budući da sinteza navedenoga statističkoga modela ne obuhvaća temu ovoga rada, nećemo analizirati statistički izračun konstanti $a$ i $b$ za slučaj Celzijeve temperature, već ćemo koristiti postejeće konstante iz literature ($a=5.085$ i $b=-0.1156$)\cite{Dekking} u svrhu izrade grafičkog prikaza vjerojatnosti kvara brtvi pri određenim temperaturama -- kao što je prikazano na slici \ref{fig:M2}.
        
        \begin{figure}[H]
            \centering
            \begin{tikzpicture}
                \datavisualization[scientific axes=clean, x axis={length=10cm, label={Temperatura spojišta (\textdegree F)}}, y axis={length=5.5cm, label={Učestalost kvarova}}]
                [
                    visualize as scatter,
                    scatter={
                        style={mark=*,mark size=1.4pt}
                    }
                ]
                data[format=table]    {
                    x, y
                    53, 3
                    57, 1
                    58, 1
                    63, 1   
                    66, 0
                    67, 0
                    67, 0
                    67, 0
                    68, 0
                    70, 0
                    70, 0
                    70, 1
                    70, 1
                    72, 0
                    73, 0
                    75, 0
                    75, 2
                    76, 0
                    76, 0
                    79, 0
                    80, 0
                    81, 0
                    83, 0
                    84, 0
                }
                [
                    visualize as line,
                    /pgf/data/evaluator=\pgfmathparseFPU %ovaj dio je zajedno sa prikladnim newcommand nužan za aproksimaciju složenih funkcija
                ]
                data[format=function]   {
                    var x : interval [30:90];
                    func y = 6*divide(exp(5.085-0.1156*\value x),(1+exp(5.085-0.1156*\value x)));
                }
                info    {
                    \draw[black] (visualization cs: x=45, y=3.5) node[above, font=\footnotesize] {$6 \cdot p(t)$};
                };
            \end{tikzpicture}
            \caption{Grafički prikaz modela \eqref{eq:U1} u odnosu na diskretne vrijednosti mjerenih podataka \cite{Dekking}}
            \label{fig:M2}
        \end{figure}


        Valjano je, temeljem prethodnih primjera, zaključiti da uspješnost ispunjavanja zadanih ciljeva uveliko ovisi o mogućnostima glavnih aktera da što brže i točnije predvide posljedice kako bi se što bolje pripremili za daljnje korake. Pouzdane prognoze upravo to i omogućuju -- da se donesu pravovremene odluke koje su temeljene na valjanim planovima. U poglavljima koja slijede, detaljnije ćemo definirati određene pojmove, pomoću kojih ćemo biti u mogućnosti lakše prevesti prognoziranje iz lingvističke apstrakcije u jezik matematike -- najprecizniji i najrašireniji jezik koji čovjek poznaje.

        %znam da mi rad vjerojatno neće pročitati više od 5 osoba, ali se trudim da bude bar malo gripping.

        \section{Osnovne vrste signala i sustava}
        %SiS_Oppenheim,_Willsky_-_Signals_and_Systems_(1997)
        %Benoit Boulet  - Fundamentals of Signals and Systems   - 1584503815
        %Praktički svih 10 knjiga koje imam sa FER2... potrebna brza analiza uvoda

            Prije matematičkoga opisa prognostičkih modela, potrebno je pobliže opisati pojam \textit{signala} kao realnu ili kompleksnu funkciju vremena. U elektrotehnici je takav opis signala od iznimne važnosti, pritom posebnu pažnju pridodajemo sinusoidnim signalima zbog karakterističnih svojstva takvih signala. Ta karakteristična svojstva pokazala su se od iznimne koristi za modeliranje fizikalnih pojava pomoću aproksimacije praktičih rezultata dobivenih pripadajućim mjernim metodama. Osim u elektrotehnici, signali su sveprisutni u gotovo svim znanstvenim disciplinama i granama.U širem smislu, valjano je reći da je signal skup podataka ili informacija, pa tako možemo konstatirati da su neki od primjera signala struja i napon, sila i brzina, tlak i protok, govor, video ili burzovni indeksi. Usprkos svojoj sveprisutnosti, signali su u suštini zadržali svoju nominalnu jednostavnost -- a to je činjenica da je signal u suštini funckija jedne ili više nezavisnih varijabli koje opisuju ponašanje promatranih fenomena. 

            %Primjer analize ponašanja signala propagacijom kroz punovalni ispravljač
            %http://www.texample.net/tikz/examples/power-electronics-rectifier/            
            
            Prema prirodi vremenske varijable, dvije najosnovnije vrste signala su:
            \begin{enumerate}
                \item \textit{vremenski kontinuirani signali:} $x(t), t\in \mathcal{R}$
                \item \textit{vremenski diskretni signali:} $x[n], n \in \mathcal{Z}$
            \end{enumerate}

            U vremenski \textit{kontinuiranim signalima} vremenska konstanta $t$ je kontinuirana i njezina domena, što rezultira i konrinuiranim vrijednostima slike promatrane funkcije. Navedeno se može grafički predočiti pomoću glatke neprekidne linije kao što je prikazano na Slici \ref{fig:M31} koja prikazuje vremenski kontinuirani signal $x(t)=sin(\frac{n\pi}{2})$. Za razliku od vremenski kontinuiranih signala, u vremenski \textit{diskretnim signalima} vremenska konstanta je definirana u točno određenim -- diskretnim -- vremenima. Posljedično, vrijednosti slike promatrane funkcije biti će poznate samo u prethodno definiranim diskretnim vremenima. Primjer vremenski diskretnog signala $x[n]=sin(\frac{n\pi}{2})$ grafički je prikazan na Slici \ref{fig:M32}.

            \begin{figure}[H]
                \centering
                \begin{subfigure}[b] {.48\linewidth}
                    \centering
                    \begin{tikzpicture}
                        \begin{axis}[
                            standard,
                            xlabel={$t$},
                            ylabel={$x(t)$},
                            enlarge x limits=false,
                            domain = -4:4,
                            xmin = -4, xmax = 4,
                            ymin = -1, ymax = 1,
                            xtick = {-4,...,4}                    
                        ],
                            \addplot [smooth, black, thick] {sin(180*x/2)};
                        \end{axis} 
                    \end{tikzpicture}
                    \caption{Vremenski kontinuirani signal}
                    \label{fig:M31}
                \end{subfigure}
                \hfill
                \begin{subfigure}[b] {.48\linewidth}
                    \centering
                    \begin{tikzpicture}
                        \begin{axis}[
                            standard,
                            xlabel={$n$},
                            ylabel={$x[n]$},
                            enlarge x limits=false,
                            %domain = -3:3,
                            %samples = 6,
                            samples at = {-4,...,4},
                            xmin = -4, xmax = 4,
                            ymin = -1, ymax = 1,
                            xtick = data,                    
                        ],
                            \addplot+[ycomb, black, thick] {sin(180*x/2)};
                        \end{axis} 
                    \end{tikzpicture}
                    \caption{Vremenski diskretni signal}      
                    \label{fig:M32}
                \end{subfigure}
                \caption{Prikaz vremenski kontinuiranoga i vremenski diskretnoga signala funkcije $sin(\frac{n\pi}{2})$}
                \label{fig:M3}    
            \end{figure}

            
            Neki od primjera vremenski kontinuiranih signala su brzina, pozicija vozila, govor ili zvuk iz audio sustava; dok su primjeri vremenski diskretnih signala mjesečne vrijednosti dionica, demografski pokazatelji poput nataliteta i sl..
            
            Kao što smo već prethodno ustanovili, vremenski diskretni signali mogu reprezentirati bilo koje promatrani fenomen čije se vrijednosti mogu dobiti za točno određeni trenutak -- što nas dovodi do zaključka o postojanju mogućnosti prikazivanja vremenski kontinuiranih signala pomoću vremenski diskretnih signala. To je moguće sukcesivnim \textit{uzorkovanjem} vremenski kontinuiranoga signala (eng. \textit{sampling}), čime dolazimo do diskretnih vrijednosti signala. Rezolucijom uzorkovanog signala moguće je upravljati pomoću podešavanja frekvencije uzorkovanja (eng. \textit{sampling rate}). Tako je prethodno navedeni diskretni signal prikazan na Slici \ref{fig:M32} načinjen uzorkovanjem kontinuiranoga signala na Slici \ref{fig:M31}.  Navedena klasa signala vrlo je korisna prilikom digitalne obrade signala. Uzmemo li u obzir prethodno spomenutu prednost visoke brzine računalnog procesuiranja podatka, ali i postojeća ograničenja računalnih sustava radnom memorijom, procesorskim performansama, te digitaliziranom reprezentacijom podataka; važnost vremenski diskretnih signala i uzorkovanja je neminovna -- što je dokazano širokom i ekonomičnom integracijom funkcionalnosti obrade signala u različita tehnološka i ina rješenja (npr. mogućnost mobilnih uređaja da prepoznaju glasovne naredbe). Naravno, operacija uzorkovanja signala nije toliko jednostavna kao što smo opisali jer kvalitetna izvedba zahtjeva dobro poznavanje spektralne analize -- što uključuje znanja iz područja matematičkih transformacija do realnih ograničenja poput preklapanja signala ili prisutnosti šuma. Analogno vrijedi i za rekonstrukciju uzorkovanih signala u vremenski kontinuirane signale.


            Uz signale se nerijetko veže i ideja \textit{sustava}, te je stoga potrebno definirati koncept sustava kao spone između ulaznoga i izlaznoga signala. Za razliku od jednostavnosti ideje signala, navedeno će možda izazvati određene poteškoće -- budući da je sustav u općem smislu shvaćen kao cjelokupno i dovršeno projektantsko rješenje (npr. sustav za prijenos električne energije), dok je matematička reprezentacija takvoga sustava predočena kao model sustava. Štoviše, opisujući sustav u užem smislu, i dalje dolazimo do vrlo širokoga polja istraživanja i unutar točno određenoga znanstvenoga područja. Zadržimo li se isključivo na području elektrotehnike, sustavi mogu biti određeni kao sustavi programske podrške, sustavi regulacije, elektronički sustavi, ugradbeni sustavi, mehanički sustavi, elektromehanički sustavi itd.. Valjano je zaključiti da je potrebno odrediti nužne granice idejnoga razmatranja sustava kako bi izbjegli neizbježnu apstrakciju beskonačne kvantizacije opisa sustava. Iz toga proizlazi ideja i potreba za prethodno spomenutim razmatranjem sustava kao poveznice između ulaznoga i izlaznoga signala, što se ispostavilo kao koristan alat inženjerima prilikom analize propagacije ulaznih signala u regulacijskim krugovima ili prilikom prognoziranja ponašanja sustava na određene pobude tijekom projektiranja samoga sustava.

            Matematički, sustavi predstavljaju modele koji repretentiraju transformaciju ulaznog signala $x(t)$ u izlazni signal $y(t)$. Navedeno se može opisati relacijom $y(t)=Hx(t)$, gdje $H$ predstavlja matematički model transformacije signala -- tj. sustav. Analogna relacija vrijedi i za slučaj transformacije vremenski diskretnoga signala $x[n]$, kao što je prikazano na Slici \ref{fig:U4}.
            
            Kao i kod signala, dvije osnovne vrste sustava prema prirodi vremenske varijable su:
            \begin{enumerate}
                \item \textit{Vremenski kontinuirani sustavi:} ulazni signal $x(t)$ i izlazni signal $y(t)$ sustava su vremenski kontinuirani
                \item \textit{Vremenski diskretni sustavi:} ulazni signal $x[n]$ i izlazni signal $y[n]$ su vremenski diskretni
            \end{enumerate}

            \begin{figure}[H]
                \centering
                \begin{subfigure}[b] {.48\linewidth}
                    \centering
                    \begin{tikzpicture}[node distance=2.5cm, auto, >=latex']
                        \centering
                        \node [int, align=center, midway] (a) {H};
                        \node (b) [left of=a,node distance=2cm, coordinate] {a};
                        \node [coordinate] (end) [right of=a, node distance=2cm]{};
                        \path[->] (b) edge node {$x(t)$} (a);
                        \path[->] (a) edge node {$y(t)$} (end);      
                    \end{tikzpicture}
                    \caption{}
                    \label{fig:U41}
                \end{subfigure}
                \hfill
                \begin{subfigure}[b] {.48\linewidth}
                    \centering
                    \begin{tikzpicture}[node distance=2.5cm, auto, >=latex']
                        \centering
                        \node [int, align=center, midway] (a) {G};
                        \node (b) [left of=a,node distance=2cm, coordinate] {a};
                        \node [coordinate] (end) [right of=a, node distance=2cm]{};
                        \path[->] (b) edge node {$x[n]$} (a);
                        \path[->] (a) edge node {$y[n]$} (end);      
                    \end{tikzpicture}
                    \caption{}
                    \label{fig:U42}
                \end{subfigure}
                \caption{Blokovski prikaz vremenski kontinuiranoga sustava (a) i vremenski diskretnoga sustava (b).}
                \label{fig:U4}
            \end{figure}

            \subsection{Kauzalnost sustava}
                Iako signali i sustavi imaju niz matematičkih svojstava koji se koriste u svakodnevnoj primjeni, u svrhu pojašnjenja budićih termina ukratko ćemo pojasniti pojam kauzalnosti sustava.


                Sukladno definiciji, sustav je \textit{kauzalan} ukoliko signal koji propagira kroz sustav ovisi isključivo o trenutnim ili prošlim vrijednostima ulaznoga signala. Drugim rječima, izlazni signal ne predviđa vrijednosti odziva na buduće vrijednosti ulaznoga signala. Konvekventno, sustav \textit{nije kauzalan} ukoliko vrijednost izlaznoga signala ovisi o budućoj vrijednosti ulaznoga signala. Primjer kauzalnog sustava je
                \begin{equation}
                    \kauzalni
                \end{equation}
                dok je
                \begin{equation}
                    \nekauzalni
                \end{equation}
                primjer sustava koji nije kauzalan.

                Iako su kauzalni sustavi vrlo važni zbog prirode realnih rješenja koja aktuiraju temeljem prethodno očitanih vrijednosti, već sada možemo naslutiti važnost nekauzalnih sustava prilikom uočavanja kretanja trendova prilikom prognoziranja ili sinteze sustava regulacije s ciljem otstranjivanja šumova i sl..

        \section{Metodologije prognoziranja}%potrebno je dodatno poraditi na naslovima i podnaslovima jer će se dalje spominjati random varijale i sl., pa je potrebno provjeriti medotologije koje su prikazane u knjizi
            Obzirom na sveprisutnost prognoziranja, baš kao i signala i sustava, klasificirati ćemo dvije osnovne vrste prognoza -- subjektivne prognoze i prognoze temeljene na statističkim modelima.

            Subjektivne prognoze temelje se na pogađanjima, prethodnim iskustvima i intuiciji. Takve prognoze nemaju predodređena i utvrđena pravila procesuiranja informacija, već se svode na donošenje zaključaka temeljem osobnog mišljenja osobe koja iznosi prognozu. Na primjer, dva će ekonomska analitičara iznesti različite prognoze za ponašanje tržišta i vrijednosti financijskih instrumenata u slučaju kriznih situacija poput pandemije koronavirusa. Dok će jedan analitičar smatrati da će vrijednosti financijskih instrumenata imati negativan trend u nadolazećem kvartalu uslijed restriktivnih mjera koje imaju nepovoljan utjecaj na tržište, drugi će prognozirati rast temeljem povećanog prilijeva kapitala od strane povećanog broja ulagača koji prvi puta ulaze na tržište kapitala uslijed straha neiskorištavanja volotilnosti tržišta za ostvarivanje dugoročnih kapitalnih dobitaka (eng. \textit{fear of missing-out}). Iako ovakve metode prognoziranja uistinu mogu rezultirati donekle točnim prognozama, kao što je to bilo u slučaju tzv. kratkih prodaja (eng. \textit{short-sell}) prilikom kraha američkog tržišta nekretnina ili prilikom razotkrivanja prikrivanja neispravnih rezultata mjerenja dušičnih oksida dizelskih automobila Volkswagen grupe; one su dokazano manje točne od prognoza temeljenih na statističkim modelima.

            
            %Objasniti na temelju par primjera i utjecaj određenih varijabli (random walk, ...)...ovo možda ipak za nešto drugo

            \textbf{(Navesti model-based: kauzalni prednosti, nekauzalni prednosti...uvod iz nekauzalnih u vremenske nizove)}

    \chapter{Vremenski niz}

        % Petrus M.T. Broersen - Automatic Autocorrelation and Spectral Analysis-Springer (2006)
        Prazan krevet kao karantena, tebe ko i sreće ima-nema. Tu tu tu tururu

        \section{Osnovni principi}
        %Signals and Systems DeMYSTiFieD A Self-Teaching Guide (dobro za elektrotehničke primjere autokorelacijske analize)
        \section{Stohastički proces}
    \chapter{Autoregresivni modeli}
        \textbf{(Pitati Sučića/Stojkovića/Bulića mogu li me usmjeriti na neke konkretne primjere iz prakse kako bi se iskoristili uz ekonometrijske primjere. Franković će vjerojatno imati dobar dataset za HEP-ova postrojenja i energetsku učinkovitost/prognoziranje potrošnje za određene trenutke kako bi se mreža prilagodila)}
        \section{Pomična srednja vrijednost (MA)}
        \section{Autoregresivni procesi (AR)}
        \section{ARMA (p,q) proces}
        \section{ARIMA proces}
    \chapter{Zaključak}
        \textbf{(Tu ide rezime core teme rada. Dodatno se može iskoristiti "Ken Holden, David A. Peel, John L. Thompson - Economic Forecasting: An Introduction-Cambridge University Press (1990)" kao outro; nadograditi vrlo dobrim dodatnim zaključcima koji su dati u konkretnom primjeru, ali poopćiti budući da nakon uvoda treba održati elektrotehniku dominantnom temom.)} % (page 29 - 33)
        \ldots{}


        Temeljem stečenih saznanja, zaključujemo da se konstrukcija modela i njihovo korištenje u prognostici može sažeti u sljedeće korake:
        \begin{enumerate}
            \item Odabir valjane teorije koja najbolje opisuje ponašanje promatranih podataka i relevantnih faktora -- koje mogu biti klasificirani kao endogeni faktori, čije je ponašanje predmet konstruiranoga modela, ili egzogeni faktori, čije je ponašanje promatrano deterministički i neovisno o egzaktnom skupu parametara određenih konstruiranim modelom.
            \item Matematički zapis korištene teorije, koji povezuje prethodno navedene faktore pomoću složene matematičke jednadžbe. Poseban naglasak pri određivanju međusobnog odnosa parametara ima rezultantni odnos prethođenja ili kašnjenja konstruiranoga modela spram varijable matematičkoga očekivanja i diskretnih mjernih podataka.
            \item Pronalazak najreprezentativnijega skupa podataka, čije su vrijednosti u skladu s mjeriteljskim normama koje podržavaju korištenu opisnu teoriju.
            \item Korištenje prikladnih kvantitativnih metoda za procjenu numeričkih vrijednosti nepoznatih parametara matematičke jednadžbe koji će najbolje aproksimirati vrijednosti u području korištenog skupa podataka.
            \item Prognoziranje budućega ponašanja predmeta konstruiranoga modela temeljem prognoziranih vrijednosti kauzalnog modela egzogenih faktora -- iz kojih nadalje proizlaze prognozirane vrijednosti endogenih faktora. Vjerodostojnost navedenoga postupka proizlazi iz poznavanja budućega ponašanja egzogenih faktora, što spriječava neželjeni efekt projekcije prošlih vrijednosti na budućnost.
        \end{enumerate}
        $$\fourierovred$$
    \begin{thebibliography}{9}
        \bibitem{NASA} Presidential Commission: "Report of the Presidential Commission on the Space Shuttle Challanger Accident", U.S. Government Printing Office, Washington, D.C., 1986.
        \bibitem{Dekking} F.M. Dekking i dr.: "A Modern Introduction to Probability and Statistics: Understanding Why and How", Springer, Sjedinjene Američke Države, 2006.
    \end{thebibliography}
\end{document}