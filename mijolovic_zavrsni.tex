\documentclass[a4paper,12pt,oneside]{memoir}

%matematički paketi

\usepackage[intlimits]{amsmath}     %omogućava postavljanje granica integrala u formulama
\usepackage{amsthm}         %matematički teoremi, leme i sl.
\usepackage{siunitx}        %podrška za korištenje SI sustava mjernih jedinica

%encoding fontova i jezika
\usepackage[croatian]{babel}
\usepackage[utf8x]{inputenc}        %encoding inputa
\usepackage[enc=utf8]{hrlatex}
\usepackage[T1]{fontenc}        %encoding fontova koji je prikazan u PDF-u
\usepackage{amsfonts}
\usepackage{dsfont}

\usepackage[fixlanguage]{babelbib}
\selectbiblanguage{croatian}
\OnehalfSpacing

%\usepackage[datetime2-croatian]{datetime2}

%paketi tablica, naslova, poglavlja i sl.
\usepackage[thinlines]{easytable}
\usepackage{tocloft}        %upravljanje izgledom tablice sadržaja
\usepackage{pdfpages}       %integracija eksternih PDF-ova
\usepackage{booktabs}       %koristi se za formatiranje tablica sukladno standardu za znanstvene radove i članke
\usepackage{indentfirst}     %dodaje tab za svaku prvu rečenicu odlomka
\usepackage{subcaption}     %koristi se za podnaslove slika, formi i sl.
\usepackage[font=it]{caption}
\captionsetup[table]{position=above}
\captionsetup[figure]{position=below}
\captionsetup{labelsep=period}

\usepackage[hidelinks]{hyperref}        %podrška za integraciju hyperlinkova
\urlstyle{same}

%grafički paketi
\usepackage{tikz}
\usetikzlibrary{datavisualization}
\usetikzlibrary{datavisualization.formats.functions}
\usepackage{pgfplots}
\usepackage{circuitikz}

\pgfplotsset{compat=1.11}
\usetikzlibrary{arrows,automata,patterns}

%misc. paketi

\usepackage{soul} %žuti marker
\usepackage{times}

%formatiranje dokumenta
\pagestyle{myheadings}
\setulmarginsandblock{2.5cm}{2.5cm}{*}
\setlrmarginsandblock{2.5cm}{2cm}{*}
\checkandfixthelayout

\setlength{\parskip}{6pt} %razmak između odlomaka

\usepackage{titlesec}   %nadomješta LaTeX makroe za naslove, odlonke, itd.

\titleformat{\chapter}
{\normalfont\fontsize{14}{14}\bfseries}
{\thechapter}
{1em}
{}
\titlespacing{\chapter}{0pt}{*4}{*1}
\titleformat{\section}
{\normalfont\fontsize{12}{14}\bfseries}
{\thesection}
{1em}
{}
\titlespacing{\section}{0pt}{*4}{*1}
\setsecnumdepth{subsection}
\maxtocdepth{subsection}
\titleformat{\subsection}
{\normalfont\fontsize{12}{14}}
{\thesubsection}
{1em}
{}
\titlespacing{\subsection}{0pt}{*4}{*1}

%formatiranje dokumenta - dodavanje točaka u naslove

\renewcommand{\thechapter}{\arabic{chapter}.}
\renewcommand{\thesection}{\thechapter\arabic{section}.}
\renewcommand{\thesubsection}{\thesection\arabic{subsection}.}

%formatiranje dokumenta - promjena fonta naslova "Sadržaj"

\renewcommand\printtoctitle[1]{\normalfont\fontsize{14}{14}\bfseries #1}

%formatiranje dokumenta - promjena riječi "Dodatak" u sadržaju

\renewcommand*{\cftappendixname}{Dodatak\space}

%formatiranje dokumenta - numeriranja tablica i slika



\renewcommand{\thetable}{\thechapter\arabic{table}}
\renewcommand{\thefigure}{\thechapter\arabic{figure}}

%teoremi

\newtheorem{theorem}{Teorem}[chapter]
\newtheorem{korolar}{Korolar}[chapter]
\newtheorem{definition}{Definicija}[chapter]
\newtheorem{lema}{Lema}[chapter]
\newtheorem{prop}{Propozicija}[chapter]
\newtheorem{primjer}{Primjer}[chapter]

%formule

\newcommand{\fourierovred}{f(x)= \frac{a_0}{2}+\sum_{n=1}^\infty (a_n cos(nx)+b_n sin(nx))}

%formatiranje dokumenta - numeriranja jednadžbi, definicija i teorema

\renewcommand\theequation{\thechapter\arabic{equation}}
\renewcommand\thetheorem{\thechapter\arabic{theorem}}
\renewcommand\thedefinition{\thechapter\arabic{definition}}
\renewcommand\thekorolar{\thechapter\arabic{korolar}}
\renewcommand\theprimjer{\thechapter\arabic{primjer}}


\author{Denis Mijolović} 

\begin{document}
    \begin{titlingpage}
        \begin{center}\linespread{1.8}\fontsize{16}{16}
            \normalfont{
                SVEUČILIŠTE U RIJECI
            }\\
            {\textbf{
                TEHNIČKI FAKULTET
                }
            }\\
        \fontsize{14}{14}\normalfont{
            Preddiplomski sveučilišni studij elektrotehnike
        }            
        \end{center}
        \vspace{6cm}
        \begin{center}\linespread{1.8}
            \fontsize{14}{14}\normalfont{
                Završni rad
            }\\
            \fontsize{16}{16}{\textbf{
                AUTOREGRESIJSKI MODELI U OBRADI SIGNALA
                }
            }
        \end{center}
        \vspace{6cm}
        \vfill\noindent
        \begin{tabular}[t]{l@{}}
            \fontsize{14}{14}\normalfont{
                Rijeka, \today
            }
        \end{tabular}
        \hfill
        \begin{tabular}[t]{l@{}}
            \fontsize{14}{14}\normalfont{
                Denis Mijolović
            }\\
            \fontsize{14}{14}\normalfont{
                0069066432
            }
        \end{tabular}
        \thispagestyle{empty}
        \newpage
            \begin{center}\linespread{1.8}
                \fontsize{16}{16}\normalfont{
                    SVEUČILIŠTE U RIJECI
                }\\
                {\textbf{
                    TEHNIČKI FAKULTET
                    }
                }\\
                \fontsize{14}{14}\normalfont{
                    Preddiplomski sveučilišni studij elektrotehnike
                }
            \end{center}
            \vspace{6cm}
            \begin{center}\linespread{1.8}
                \fontsize{14}{14}\normalfont{
                    Završni rad
                }\\
                \fontsize{16}{16}{\textbf{
                    AUTOREGRESIJSKI MODELI U OBRADI SIGNALA
                    }
                }\\
                \fontsize{14}{14}\normalfont{
                Mentor: doc. dr. sc. Ivan Dražić
            }\\
            \fontsize{14}{14}\normalfont{
                Komentor: prof. dr. sc. Viktor Sučić
            }
            \end{center}
        \vfill\noindent
        \begin{tabular}[t]{@{}l}
            \fontsize{14}{14}\normalfont{
                Rijeka, \today
            }        
        \end{tabular}
        \hfill
        \begin{tabular}[t]{@{}l}
            \fontsize{14}{14}\normalfont{
                Denis Mijolović
            }\\
            \fontsize{14}{14}\normalfont{
                0069066432
            }
        \end{tabular}
        \thispagestyle{empty}
        \newpage
            \thispagestyle{empty}
            Na mjesto ovo stranice je potrebno umetnunti izvornik zadatka
        \newpage
            \thispagestyle{empty}
                \begin{center}\linespread{1.8}\fontsize{16}{16}
                    {\textbf{
                        IZJAVA
                        }
                    }
                \end{center}
            \vspace{5cm}
            Sukladno članku 8. Pravilnika o završnom radu, završnom ispitu i završetku prediplomskih sveučilišnih studija/stručnih studija
			Tehničkog fakulteta Sveučilišta u Rijeci od 1. veljače 2020., izjavljujem da sam samostalno izradio završni rad
            prema zadatku preuzetom dana 18. ožujka 2019. godine.
            \vspace{3cm}


            \begin{tabular}[t]{@{}l}
                \normalfont{
                    Rijeka, \today
                }
            \end{tabular}
            \hfill
            \begin{tabular}[t]{@{}l} 
                \fontsize{14}{14}\hrulefill\\
                \fontsize{12}{12}\normalfont{
                    \phantom{Raz}Denis Mijolović\phantom{Raz}
                }
            \end{tabular}
            \newpage
                \thispagestyle{empty}
                $ $
                    \vspace{5cm}
                    \textit{
                        \\
                        Lorem ipsum dolor sit amet, consectetur adipiscing elit. Mauris et ante a mauris consequat tempor. Donec vulputate, nulla pharetra finibus finibus, ipsum ante molestie mauris, in tincidunt libero augue sit amet arcu. In hac habitasse platea dictumst. Nulla sagittis ante ac augue volutpat cursus. Ut gravida odio malesuada, vestibulum mi at, dignissim leo. Morbi semper finibus est, id maximus mauris auctor eget.
                    }
    \end{titlingpage}
    \begin{KeepFromToc}
        \tableofcontents
    \end{KeepFromToc}
    \chapter{Uvod}
        \textbf{(Definicija podataka i informacija)}

        Vratimo li se u prošlost, važnost podataka i njihova inerpretacija u svrhu stvaranja informacija bila je neminovna za civilizacijske strukture koje danas poznajemo. Od početka razvoja primitnivnih tehnologija i njezinoga korištenja za optimizaciju životnih procesa, do razvoja trgovine i financijskih instrumenata; čovjek je sve efikasnije koračao prema evolucijskom vrhu uspješnom utilizacijom svoje okoline. Daljnjim razvojem, čovjek je nadišao barijeru vlastite memorije za obradu podataka, te se počeo približavati točki zasićenja - uslijed koje se javila potreba za stvaranjem boljega i efikasnijega načina obrade podataka i njihove pretvorbe u sažetiju i jednako kvalitetnu informaciju.

        Kao i kod većine velikih znanstvenih otkrića koja su obilježila svoje razdoblje kao pretkretnice stoljeća u ratnim okolnostima, tako je i tijekom Drugog svjetskog rata otkriven novi izum kao odgovor na Enigmu - stroj koji je kodirao njemačke vojne i diplomatske poruke te čija se šifra smatrala nerješivom. Uzevši u obzir širenje nacističke Njemačke i prijetnju koja je prijetila ljudskim slobodama, javila se jasna potreba za stvaranjem rješenja koje će nadići čovjekova fizička i psihička ograničenja te biti u mogućnosti procesuirati veliku količinu podataka u vrlo kratkom vremenu -- radeći neprestano i ne osjećajući umor te, konzekventno, ne stvarajući greške. Imajući to na umu, Englezi su tada izumili \textit{Colossus} -- uređaj koji je pomogao kriptoanalitičarima da dešifriraju njemačke poruke i dao im stratešku prednost pri prognozorianju daljnih koraka tijekom ratnog razdoblja. Taj uređaj se u današnje vrijeme smatra pretečom prvoga računala kakvim ga danas poznajemo.

        Baš kao i tada, te kroz cijelu ljudsku povijest, ispravna interpretacija podataka u informaciju koju krajnji korisnik može iskoristiti u svrhu kvalitetnijeg donošenja važnih odluka, iskazala se kao značajan faktor u očuvanju civilizacijske i sveopće stabilnosti -- stanja koje je najpotentnije za budući razvoj, a time i blagostanje. Kako je civilizacija težila ka tehnološkom društvu, tako je eksponencijalno rasla i količina podataka koju je bilo potrebno interpretirati; te se javila potreba za sve preciznijim prognoziranjem budućega ponašanja sustava kako ne bi došlo do destrukcije vitalnih procesa i porasta volatilnosti sustava - a samim time i povećanjem vjerojatnosti za njegovim urušavanjem.

        Slijedno tome, važnost prognoziranja proizlazi iz dvije osnovne činjenice:
        \begin{enumerate}
            \item budućnost je nepredvidiva
            \item poduzete akcije u vrijeme donošenja odluke vrlo često nemaju trenutne posljedice sve do određenoga trenutka u budućnosti
        \end{enumerate}

        Imajući na umu prethodno navedene činjenice, slijedno je zaključiti da precizne prognoze budućih događaja pospješuju efikasnost postupka donošenja odluka. Štoviše, većina odluka od posebnog poslovnog ili političkog značaja imaju nužan uvjet pozitivnih prognostičkih indikatora prezentiranih kroz studije izvedivnosti ili ispitivanja javnoga mnijenja -- čija je zadaća što preciznije prognozirati buduću potražnju za proizvodima i uslugama koje su predmet proizvodnih, industrijskih ili inih procesa. U konkretnijem, mikroekonomskom smislu, dalo bi se konstatirati da je prognoziranje implementirano u gotovo sve odluke koje su dio naše svakodnevice: poduzeće će početi izgradnju nove proizvodne jedinice kako bi osiguralo rastuću potražnju na tržištu; zaposleni radnici početi će štedjeti kako bi si mogli priuštiti godišni odmor ili kupnju vrijednosnica s ciljem ostvarivanja prava na buduću kapitalnu dobit; rektor sveučilišta donesti će odluku o otvaranju novog studijskog programa kako bi povećao broj studenata i uskladio broj obrazovanih stručnjaka s budućim tržišnim potrebama. Ukratko, svaki od navedenih procesa zahtjeva inicijalnu percepciju utjecaja donešenih odluka na povećanje vjerojatnosti ostvarivanja zadanih ciljeva -- i to prije donošenja samih odluka.


        Ostanemo li na mikroekonomskoj razini, možemo reći da je poslovanje poduzeća zahvaćeno trijema faktorima: makroekonomskim prilikama, industrijskim pokazateljima i samim uređenjem poduzeća. Obzirom da direktor takvoga poduzeća najčešće ima mogućnost direktnoga djelovanja jedino na poslijednja dva faktora, od iznimne je važnosti da bude što bolje informiran o vanjskim utjecajima i trendovima na koje ne može izravno djelovati -- poput donošenja novih zakonskih regulativa ili razvoja globalne virusne pandemije. Važnost ispravnog prognoziranja čimbenika na koje se nema direktan utjecaj od iznimne je važnosti za većinu projekta iznimno velike vrijednosti. Tako, na prijmer, projekti s velikim početnim ulaganjima poput izgradnje novih jedinica za opskrbu električnom energijom, mogu imati vrijeme izrade studije izvedivosti u trajanju od 10 ili više godina prije nego što li se započnu radovi. Budući da opskrba električnom energijom ovisi o stupnju razvijenosti regije ili područja, potrebno je izraditi valjane prognoze koje će pomoći u procjeni trenutne, kratkoročne i dugoročne isplativosti novih opskrbnih jedinica za opskrbljivača, ostale projektne partnere, te za lokalni i regijalni razvoj. U tom slučaju, prisutnost raznovrsnih rizika utječe na sve dionike projekta -- od nacionalne, regionalne ili lokalne uprave ili samouprave; do svih industrijskih poduzeća koja se nalaze na tom području. Ti rizici mogu biti smanjenje likvidnost tržišta, kreditni status\footnote{Usporedba potraživanja i dugovanja nekoga poduzeća koje se dalje koriste za utvrđivanje kreditnog rejtinga.}, kamatni rizik\footnote{Moguće smanjenje vrijednosti financijskih instrumenata javno dostupnoga poduzeća uslijed promjene razine kamatnih stopa na tržištu.},valutni rizik\footnote{Osjetljivost financijskih instrumenata na fluktuacije tečajeva stranih valuta. Značajno za multinacionalna poduzeća s viševalutnim potraživanjima i dugovanjima.}, strateški rizik\footnote{Rizik za prihode ili kapital koji nastaje kao posljedica neadekvatnih poslovnih odluka ili nepravilnog vođenja poduzeća.}, stanje tržišta rada i još mnogi drugi. Pritom je važno napomenti važnost osjetljivosti modela prognoziranja na fluktuacije u svakoj stavci procjene ukupnoga rizika, tj. isti prognostički model se ne bi trebao koristiti za prognoziranje u različitim zakonskim i regulativnim okvirima -- što pridonosi kompleksnosti izrade samoga modela i porastu troškova izrade kvalitetne studije.


        U navedenim primjerima nastojali smo nešto konkretnije opisati važnost prognoziranja kroz opće primjere iz prakse. Kako bismo zaokružili sadržaj ovoga poglavlja, osvrnuti ćemo se na stavrni primjer jedne od najutjecajnijih nesreća koje su obilježile prošlo stoljeće --  no ovoga puta fokusirajući se na ispravnoj interpretaciji prognostičkih procesa i važnosti najnasumičnijeg čimbenika u cijelome procesu odlučivanja, tj. čovjeka.


        28. siječnja 1986. godine, svemirska letjelica \textit{Challanger} eksplodirala je otprilike 70 sekundi nakon polijetanja iz svemirskog centra John F. Kennedy u Floridi. Sedmero astronauta je poginulo, a svemirska letjelica u potpunosti je uništena. Uzrok nesreće bila je eksplozija spremnika za raketno gorivo, prouzrokovana zapaljenjem plina u pomoćnim raketama za uzlijetanje.

        Pomoćne rakete za uzlijetanje bile su predmet zabrinutnosti još pri samim počecima razvoja aeronautike. Proizvodnja istih realizira se spajanjem više manjih segmenata uz veliki broj spojišta koji moraju zadovoljavati niz projektom predodređenih uvjeta -- među kojima je postizanje nepropusnosti spojišta brtvljenjem. U ovom konkretnom slučaju, korištene su meke profilne brtve s elastičnim deformacijama (također poznate kao o-prsteni) uz sloj ljepila. Prilikom zapaljenja raketnog motora, u sustavu se stvaraju visoke temperature i visoki tlak -- što utječe na otapanje ljepila i posljedično habanje brtve, te propuštanja eksplozivnog medija.

        Nakon eksplozije, istražni stožer objavio je izvješće u kojemu su navedene okolnosti koje su dovele do uzroka nesreće. Utvrđeno je kako je prethodne večeri, 27. siječnja, donesena odluka o lansiranju letjelice sljedećega dana usprkos nepovoljnoj prognozi temperaturnih uvjeta određenih radnom temperaturom deklariranom od strane proizvođaća pomoćnih raketa. Protivno apelu inženjera za obustavom lansiranja, odluka uprave bila je da će se lansiranje ipak provesti, što je rezultiralo kobnim posljedicama.


        Kako bi izračunali vjerojatnost zakazivanja brtvi, koristiti ćemo podatke zakazivana brtvi prilikom prethodnih lansiranja -- budući da je lansiranje \textit{Challanger-a} bilo je 24. lansiranje aktualnog svemirskoga programa. Budući da su za svaku pomoćnu raketu korištena tri o-prstena te da su za dotadašnja lansiranja korištene dvije pomoćne rakete po lansiranju, razmatrati ćemo korištenje 6 o-prstena po lansiranju. Uslijed zavisnosti karakteristika brtve o temperaturi, razmotriti ćemo ovisnost zakazivanja brtve o temperaturi koline prilikom lansiranja.

        %https://history.nasa.gov/rogersrep/v1p146.htm
        %DTIC_ADA171402 (page 151) --> temperature na page 136
        %F.M. Dekking, C. Kraaikamp, H.P. Lopuhaä, L.E. Meester-A Modern Introduction to Probability and Statistics (page 20)
        \begin{center}
        \begin{tikzpicture}
            \draw[ultra thin, color=lightgray] (15,3) grid (0,0);
            \draw[-] (0,0) -- (15,0) node[below=0.6cm] {$t/\textit{\textdegree C}$};
            %\node[anchor=0.6cm, align=center] {$\text{tekst}$};
            \node[anchor=north] at (0,0) {$5$};
            \node[anchor=north] at (3,0) {$10$};
            \node[anchor=north] at (6,0) {$15$};
            \node[anchor=north] at (9,0) {$20$};
            \node[anchor=north] at (12,0) {$25$};
            \node[anchor=north] at (15,0) {$30$};
            %podijeliti celzijuse sa 1.67 kako bi se dobile koordinate na x-osi (popraviti cijeli tikz ispod ove linije)
            
            \draw[-] (0,0) -- (0,3);
            \node[anchor=east] at (0,0) {$0$};
            \node[anchor=east] at (0,1) {$1$};
            \node[anchor=east] at (0,2) {$2$};
            \node[anchor=east] at (0,3) {$3$};
            \filldraw
                (3.556,3) circle (2pt) node[align=center, above] {STS 51-C}
                (4.171,1) circle (2pt) node[align=center, above] {41B}
                (4.338,1) circle (2pt) node[align=center, below] {61C}
                (5.171,1) circle (2pt) node[align=center, above] {41C}
                (6.34,1.05) circle (2pt) node[align=center, above] {41D}
                (6.34,0.95) circle (2pt) node[align=center, below] {STS-2}
                (9.167,2) circle (2pt) node[align=center, above] {61A}

                %successful launches
                (5.6723,0) circle (2pt)
                (5.838,0) circle (2pt)
                (5.838,0.1) circle (2pt)
                (5.838,0.2) circle (2pt);
        \end{tikzpicture}

        \begin{tikzpicture}
            \datavisualization [scientific axes=clean, visualize as smooth line]
                data [format=function] {
                    var x : interval [0:5];
                    func y = \value x * \value x;
                };
        \end{tikzpicture}
        \end{center}
        \textbf{(Unesti podatke iz fig. 1.3. i pogledati mogućnosti izrade grafova u TeX-u. Preračunati u \textdegree C)}

        \textbf{(Navesti metode prognoziranja (subjective, model-based) i podjelu modela na kauzalne i nekauzalne -- uvod u novo poglavlje o teoriji signala i sustava. Možda rastaviti sljedeći paragraf i ubaciti ovaj tekst između.)}

        % F.M. Dekking, C. Kraaikamp, H.P. Lopuhaä, L.E. Meester-A Modern Introduction to Probability and Statistics


        Valjano je, temeljem prethodnih primjera, zaključiti da uspješnost ispunjavanja zadanih ciljeva uveliko ovisi o mogućnostima glavnih aktera da što brže i točnije predvide posljedice kako bi se što bolje pripremili za daljnje korake. Pouzdane prognoze upravo to i omogućuju -- da se donesu pravovremene odluke koje su temeljene na valjanim planovima. U poglavljima koja slijede, detaljnije ćemo definirati određene pojmove, pomoću kojih ćemo biti u mogućnosti lakše prevesti prognoziranje iz lingvističke apstrakcije u jezik matematike -- najprecizniji i najrašireniji jezik koji poznajemo.

        \section{Teorija signala i sustava}
            Napraviti uvod o stručnoj terminologiji (kauzalni, nekauzalni signali), a koji se ne tiče prestrikno više matematike.
            \subsection{Vrste signala}
            \subsection{Vrste sustava}
        \section{Metodologije prognoziranja} %potrebno je dodatno poraditi na naslovima i podnaslovima jer će se dalje spominjati random varijale i sl., pa je potrebno provjeriti medotologije koje su prikazane u knjizi
            Objasniti na temelju par primjera i utjecaj određenih varijabli (random walk, ...)

            Započeti uvod prema vremenskim nizovima, a zatim povezati korištene metodologije sa matematičkim modeliranjem.
    \chapter{Vremenski niz}

        % Petrus M.T. Broersen - Automatic Autocorrelation and Spectral Analysis-Springer (2006)
        Prazan krevet kao karantena, tebe ko i sreće ima-nema. Tu tu tu tururu

        \section{Osnovni principi}
        \section{Stohastički proces}
    \chapter{Autoregresivni modeli}
        \section{Pomična srednja vrijednost (MA)}
        \section{Autoregresivni procesi (AR)}
        \section{ARMA (p,q) proces}
        \section{ARIMA proces}
    \chapter{Zaključak}
        \ldots{}Ovo je neki tekst za kraj. To jest, drugi section.
        $$\fourierovred$$
\end{document}